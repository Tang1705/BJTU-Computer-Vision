\IEEEraisesectionheading{\section{简介}
\label{sec:introduction}}

模板匹配是通过一张模板图片去另一张图中找到与模板相似部分的一种算法。一般是通过滑窗的方式在待匹配的图像上滑动,通过比较模板与子图的相似度,找到相似度最大的子图。这种算法最核心部分在于如何设计一个相似性度量函数。一般最容易想到的相似性度量函数便是欧式距离。

但有许多因素会使得模版匹配带来巨大的开销,例如场景图像的大小、模板的数量和角度等。针对不同的因素,我们可以采用不同的方法来减少计算开销。例如,可以利用图像金子塔来同时降低模板图像和场景图像的分辨率,并在较低的分辨率进行匹配,进而在更高的分辨率逐步优化;或是利用模版图像的边缘和场景图像的边缘来进行匹配。对于后者而言,欧式距离显然已不再适合对图像边缘之间的相似性进行度量,因而需要设计一个新的相似性度量函数。

Hausdorff 距离则被用于描述两组点集之间相似程度,是两个点集之间距离的一种定义形式。通俗来讲,Hausdorff 距离描述的是一个集合到另一个集合中最近点的最大距离。但 Hausdorff 距离本质上并没有减少匹配过程中的计算开销。为了实现更高效的搜索和匹配,可以考虑先对场景图像进行距离变换,即利用切比雪夫距离(Chebyshev Distance)预先计算生成边缘的距离图像,以此达到降低匹配复杂度的目的。

本文剩余部分将相关匹配、基于 Hausdorff 距离匹配以及对场景图像距离变换的 Hausdorff 距离匹配三种方法的原理、实现及实验进行介绍,所有实验涉及的方法、函数实现均基于 Python 语言。