\IEEEtitleabstractindextext{
\begin{abstract}
\justifying 在运动目标检测提取中,背景目标对于目标识别和跟踪至关重要。图像变化的原因一般分成相机运动和相机固定。前者一般用光流法,后者一般用背景建模法。基于高斯混合模型的背景建模适合于在相机固定的情况下,从图像序列中分离出背景和前景。其在物体具有重复性运动的情况下鲁棒性较好,比如微风树叶抖动。本文将对基于高斯混合模型的背景建模方法进行介绍,其效果及实时性分析在实验部分给出。

\end{abstract}

\begin{IEEEkeywords}
计算机视觉、背景建模、高斯混合模型
\end{IEEEkeywords}}