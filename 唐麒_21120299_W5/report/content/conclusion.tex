\section{总结}
本文主要对基于高斯混合模型的背景建模方法进行了介绍,并对不同实现方法的建模效果及实时性进行了实验和分析。基于高斯混合模型的背景建模方法具有更强的鲁棒性和准确性,通过多个高斯模型加权构建实时背景,更贴近实际场景;同时减少了光线变化、空气流动及相机抖动等复杂因素的干扰,避免了对新加入后静止不动的目标的误判发生。一定程度上减少了背景差分法对背景的依赖。通过膨胀、腐蚀等处理,可以得到更加清晰和完整的运动目标。在实时性方面,仅在灰度图像上对每个像素建立高斯混合模型,可以取得性能和实时性的较好平衡,而 OpenCV 封装的高斯混合模型函数的实时性最好。