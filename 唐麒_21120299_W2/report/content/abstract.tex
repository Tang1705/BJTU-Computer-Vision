\IEEEtitleabstractindextext{
\begin{abstract}
\justifying 纹理作为图像的一个重要属性,在计算机视觉和图像处理中占有举足轻重的地位。对于纹理的分析可以用于图像合成、图像匹配等任务。纹理作为一种区域特性,与区域的大小和形状有关。两种纹理模式之间的边界,可以通过观察纹理度量是否发生显著改变来确定。因此分析纹理可以得到图像中物体的重要信息,是特征提取、图像分割的重要手段。纹理分析主要包括纹理表征和纹理分割两个任务,本文利用灰度共生矩阵和 Gabor 滤波器两种方法对合成纹理图像的纹理进行描述,从而形成特征向量,并进一步利用无监督聚类方法对特征向量空间中的点进行聚类,最终完成对纹理图像的分割。纹理特征的提取及其分割结果将在本文的实验部分给出,不同纹理表征方法和部分参数设置对实验结果的影响也会进行简单探讨。


\end{abstract}

\begin{IEEEkeywords}
计算机视觉、纹理图像、图像分割、灰度共生矩阵、Gabor 变换、聚类
\end{IEEEkeywords}}