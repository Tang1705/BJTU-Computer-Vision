\IEEEraisesectionheading{\section{简介}
\label{sec:introduction}}

图像分割是指根据灰度、彩色、空间纹理、几何形状等特征把图像划分成若干个互不相交的区域,使得这些特征在同一区域内表现出一致性或相似性,而在不同区域间表现出明显的不同。简而言之就是在一副图像中,把目标从背景中分离出来。对于灰度图像来说,区域内部的像素一般具有灰度相似性,而在区域的边界上一般具有灰度不连续性。 

近年来,随着硬件设备的不断提升和数据量的大幅度增长,深度学习技术在相当多的计算机视觉任务上都取得了令人瞩目的成绩,包括图像分割任务。而在此之前,研究者利用数字图像处理、拓扑学、数学等方面的知识来进行图像分割的方法,主要包括:基于阈值的分割方法、基于聚类的分割方法、基于区域的分割方法、基于图论的分割方法和基于边缘的分割方法等。

本文所采用的实验方法为基于阈值的分割方法。这类方法的核心思想是基于图像的灰度特征来计算一个或多个灰度阈值,并将图像中每个像素的灰度值与阈值相比较,最后将像素根据比较结果分到合适的类别中。因此,该类方法最为关键的一步就是按照某个准则函数来求解最佳的灰度阈值。

基于阈值的分割方法适用于目标和背景占据不同灰度范围的图像。理想情况下,目标和背景在其所属区域内具有恒定的不同灰度,则只需选取一个阈值便可对图像进行分割。但实际上,由于拍摄时的噪声、非均匀照明和目标表面对光照的非均匀反射等因素的干扰,其灰度值并不是恒定的,因此固定阈值分割难以有出色表现。这种情况下,我们可以根据不同图像的特性分别采用不同的阈值进行分割,即自适应阈值分割。

基于阈值的分割方法大多利用了图像的灰度直方图,它很好地反映了一幅图像中的灰度分布信息,是阈值选取的重要参考依据。在对直方图区域进行划分后,可以得到相应的直方图统计信息(目标/背景的先验概率、均值等),由此可以进一步求解不同的阈值选取准则函数。

本文剩余部分将主要对基于直方图的自适应阈值分割方法的原理、实现及实验进行介绍,所有实验涉及的方法、函数实现均基于 Python 语言。
