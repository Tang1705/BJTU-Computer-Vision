\section{总结}
本文主要的对与图像纹理相关的两个任务进行了探讨和实验,即纹理表征和纹理分割。相比较基于阈值的方法将图像分割为目标和背景两部分,基于聚类技术的纹理图像分割任务更加具有挑战性,其难点在于前期的纹理分析,如何较好地构建纹理特征,将在很大程度上对后续的纹理分割有影响。本文分别采用了灰度共生矩阵和 Gabor 滤波器两种方法分析图像的纹理特征。灰度共生矩阵方法简单,易于实现,但无法利用全局信息,与人类视觉模型不匹配,且计算复杂度较高,计算耗时。而 Gabor 滤波器具有较好的时-频局部话特性,其频率和方向与人类的视觉系统类似,特别适合于纹理表征与分割。
