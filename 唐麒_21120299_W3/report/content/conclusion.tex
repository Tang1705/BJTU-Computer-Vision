\section{总结}
本文主要对模版匹配的三种方法进行了探讨和实验,即相关匹配方法、Hausdorff 距离匹配方法和图像距离变换方法。在对模版图像进行归一化处理后,相关匹配方法的定位精度略高于 Hausdorff 距离匹配方法,这是由于在边缘检测时可能存在的噪声点对 Hausdorff 距离的计算影响较大,从而会影响到后续的定位精度。而对于定位效率而言,图像距离变换方法的定位效率远高于 Hausdorff 距离匹配方法,这是由于 Hausdorff 距离匹配方法只是一种点集之间的相似性度量,实质上并没有减少运算量,而图像距离变换方法可以预先计算出图像中每个位置与边缘点的距离,从而在匹配过程中仅需查找即可得到匹配需要的距离,以此减少运算量,提升定位效率。