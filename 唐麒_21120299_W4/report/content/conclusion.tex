\section{总结}
本文主要对目标跟踪的两种方法进行了探讨和实验,即基于均值的目标跟踪方法和基于分类思想的目标跟踪方法。前者计算量不大,在目标区域已知的情况下完全可以做到实时跟踪,但相对而言对边缘遮挡、目标旋转、变形和背景运动不敏感。而后者将背景信息考虑在内,将目标跟踪看作分类问题,把耗费时间的训练阶段分解成一系列简单的可以在线的学习计算任务,同时自动调整了不同分类器的权值,在不同的特征值空间训练,可以保持稳定的变化来适应光照和遮挡等。但由于粗糙的特征标注和缺乏尺度的变化,在实际实验中稳定性和实时性还有待进一步提高。