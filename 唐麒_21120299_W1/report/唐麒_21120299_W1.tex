\documentclass[10pt,journal,compsoc]{IEEEtran}

\usepackage{ctex}
\usepackage{times}
\usepackage{epsfig}
\usepackage{graphicx}
\usepackage{amsmath}
\usepackage{amssymb}
\usepackage{hyperref}
\usepackage{enumerate}
\usepackage{enumitem}
\usepackage{caption}
\usepackage{color}
\usepackage{comment}
\usepackage{url}
\usepackage{xcolor}
\usepackage{tabu}
\usepackage{booktabs}
\usepackage{makecell}
\usepackage{wrapfig}
\usepackage{breakcites}
\usepackage{subfig}
\usepackage{ragged2e}
\usepackage{stfloats}
\usepackage{xcolor}
\usepackage[export]{adjustbox}
\usepackage{indentfirst}
\setlength{\parindent}{2em} 

\usepackage{algorithm}
\usepackage[noend]{algpseudocode}

% 定义代码样式
\usepackage{listings}

\definecolor{gray}{rgb}{0.96,0.96,0.96}

\lstset{ %
  language=python,                % the language of the code
  basicstyle=\footnotesize,           % the size of the fonts that are used for the code
  numbers=left,                   % where to put the line-numbers
  columns=fixed, 
  numberstyle=\tiny\color{black},  % the style that is used for the line-numbers
  stepnumber=1,                   % the step between two line-numbers. If it's 1, each line 
                                  % will be numbered
  numbersep=1.5mm,                  % how far the line-numbers are from the code
  xleftmargin=1.3em,
  backgroundcolor=\color{gray},      % choose the background color. You must add \RequirePackage{color}
  showspaces=false,               % show spaces adding particular underscores
  showstringspaces=false,         % underline spaces within strings
  showtabs=false,                 % show tabs within strings adding particular underscores
  frame=single,,                 % adds a frame around the code
  frameround = tttt,
  framexleftmargin=3mm, 
  rulecolor=\color[RGB]{158,193,243},        % if not set, the frame-color may be changed on line-breaks within not-black text (e.g. commens (green here))
%  aboveskip=1em,
  tabsize=2,                      % sets default tabsize to 2 spaces
  captionpos=b,                   % sets the caption-position to bottom
  breaklines=true,                % sets automatic line breaking
  extendedchars=false,   
  breakatwhitespace=false,        % sets if automatic breaks should only happen at whitespace
  title=\lstname,                   % show the filename of files included with \lstinputlisting;
                                  % also try caption instead of title
  keywordstyle=\color[RGB]{0,51,179},          % keyword style
  commentstyle=\color[RGB]{140,140,140},       % comment style
  stringstyle=\color[RGB]{6,125,23},         % string literal style
  identifierstyle=\color{black},
  escapeinside={\%*}{*)},            % if you want to add LaTeX within your code
  morekeywords={*,...}               % if you want to add more keywords to the set
}

\def\Plus{\texttt{+}}

\makeatletter
\def\BState{\State\hskip-\ALG@thistlm}
\makeatother

\graphicspath{{figures/}}

% *** CITATION PACKAGES ***
%
\ifCLASSOPTIONcompsoc
  % IEEE Computer Society needs nocompress option
  % requires cite.sty v4.0 or later (November 2003)
  \usepackage[nocompress]{cite}
\else
  % normal IEEE
  \usepackage{cite}
\fi

% *** GRAPHICS RELATED PACKAGES ***
%
\ifCLASSINFOpdf

\else

\fi

\hyphenation{op-tical net-works semi-conduc-tor}


\begin{document}
\begin{sloppypar}

\title{计算机视觉作业­1:基于直方图的自适应阈值分割}

\author{唐麒\qquad 2110299\\qitang@bjtu.edu.cn}

\IEEEtitleabstractindextext{
\begin{abstract}
\justifying 模板匹配是一项在一幅图像中寻找与另一幅模板图像最匹配(相似)部分的技术。本文分别采用相关匹配 (Correlation Matching)、基于 Hausdorff 距离匹配以及对场景图像距离变换 (Distance Transform) 的 Hausdorff 距离匹配三种方法对给定的模板图像在场景图像中进行定位。不同方法的定位精度和定位效率将在实验部分进行简单探讨。


\end{abstract}

\begin{IEEEkeywords}
计算机视觉、模板匹配、相关匹配、Hausdorff 距离
\end{IEEEkeywords}}


% make the title area
%\maketitle

\twocolumn[{%
\renewcommand\twocolumn[1][]{#1}%
\maketitle
\pagenumbering{arabic}

\vspace{-1.5cm}
\noindent\begin{minipage}{\linewidth} 
 	\begin{center}
 	\captionsetup{font=small}
 	\begin{tabular}{@{}c@{}c@{}c@{}}
% 	\includegraphics[width=0.31\linewidth,cframe=red!50!black 1mm]{fig1_a} &
% 	\includegraphics[width=0.31\linewidth,cframe=green!50!black 1mm]{fig1_b} &
%    \includegraphics[width=0.313\linewidth,cframe=blue!50!black 1mm]{fig1_c} \vspace{-1mm}\\
	\includegraphics[width=0.31\linewidth]{fig1_a} &
 	\includegraphics[width=0.31\linewidth]{fig1_b} &
    \includegraphics[width=0.32\linewidth]{fig1_c} \vspace{-1mm}\\
    {\small (a)} &  {\small (b)}  &  {\small (c)} \\
    \end{tabular}
	\vspace{-3mm}
	\captionof{figure}{\small 原始图像及三个不同阈值下的分割结果: (a) Test\_Img\_1, (b) Test\_Img\_2, and (c) Test\_Img\_3. 三个阈值分别为125(右上)、99(左下)和156(右下)}
	\label{fig:random_thresholding}
	\end{center}  \vspace{1.25cm}
\end{minipage}
}]

\IEEEraisesectionheading{\section{简介}
\label{sec:introduction}}

在计算机视觉众多的技术领域中,目标检测(Object Detection)也是一项非常基础的任务,图像分割、物体追踪、关键点检测等通常都要依赖于目标检测。在目标检测时,由于每张图像中物体的数量、大小及姿态各有不同,也就是非结构化的输出,这是与图像分类非常不同的一点,并且物体时常会有遮挡截断,所以物体检测技术也极富挑战性,从诞生以来始终是研究学者最为关注的焦点领域之一。

在计算机视觉中,图像分类、目标检测和图像分割都属于最基础、也是目前发展最为迅速的3个领域,如图 \ref{fig:cv_task} 所示,这几个任务之间的区别为:

\begin{itemize}
	\item 图像分类:输入图像往往仅包含一个物体,目的是判断每张图像是什么物体,是图像级别的任务,相对简单,发展也最快。
	\item 目标检测:输入图像中往往有很多物体,目的是判断出物体出现的位置与类别,是计算机视觉中非常核心的一个任务。
	\item 图像分割:输入与物体检测类似,但是要判断出每一个像素属于哪一个类别,属于像素级的分类。图像分割与物体检测任务之间有很多联系,模型也可以相互借鉴。
\end{itemize}

在利用深度学习做物体检测之前,传统算法对于目标检测通常分为3个阶段:区域选取、特征提取和体征分类。首先选取图像中可能出现物体的位置,由于物体位置、大小都不固定,因此传统算法通常使用滑动窗口(Sliding Windows)算法,但这种算法会存在大量的冗余框,并且计算复杂度高。在得到物体位置后,通常使用人工精心设计的提取器进行特征提取,如 SIFT 和 HOG 等。由于提取器包含的参数较少,并且人工设计的鲁棒性较低,因此特征提取的质量并不高。最后,对上一步得到的特征进行分类,通常使用如SVM、AdaBoost的分类器。

\begin{figure}[!ht]
  \centering
  \includegraphics[width=\linewidth]{fig1}
  \caption{计算机视觉中的图像分类、目标检测和图像分割任务示意图}
  \label{fig:cv_task}
  \vspace{-0.5cm}
\end{figure}

基于深度学习的目标检测方法逐渐使目标检测进入到快速发展的阶段,比较流行的算法可以分为两类,一类是基于 Region Proposal 的R-CNN系算法(RCNN、SPPNet、FasterRCNN、Pyramid NetWorks等),它们是two-stage的,需要先算法产生目标候选框,也就是目标位置,然后再对候选框做分类与回归。而另一类是 Yolo,SSD 这类 one-stage 算法,其仅仅使用一个卷积神经网络直接预测不同目标的类别与位置。第一类方法是准确度高一些,但是速度慢,但是第二类算法是速度快,但是准确性要低一些。

本文分别利用传统的基于滑窗的目标检测算法(即机器学习方法)实现静态场景下的侧视车辆检测和基于深度学习的目标检测模型对口罩进行检测。本文剩余部分将对传统目标检测放法的原理和采用的深度学习模型的架构、实现及实验进行介绍,所有实验涉及的方法、函数实现均基于 Python 语言,其中深度学习模型的实现及训练等采用了开源深度学习框架 PyTorch。
\section{方法} 
\label{sec:proposed}

这一章节将对基于 Mean Shift 和基于分类思想的两种目标跟踪方法进行介绍,其结果将在实验部分给出。

\subsection{目标跟踪的基本框架}

跟踪算法通常同时依赖外观建模和运动信息建模。如图 \ref{fig:tracker_example} 所示,通过利用初始帧的目标外观信息进行建模后,跟踪器具有了目标和背景的辨别能力。在后续帧中,跟踪算法首先通过运动模型,如粒子滤波,粗略地估计目标位置并得到一系列的候选样本,进一步结合外观模型进行目标的精准定位。准确的定位用于更新运动模型和外观模型。因此,两种目标跟踪方法的初始目标位置标定都依赖于手工标定,后续帧中的目标位置则分别通过均值漂移方法和在线分类器得到。

\subsection{基于均值漂移的目标跟踪}

均值漂移 (Mean Shift) 的基本思想是利用概率密度的梯度爬升来寻找局部最优,即 Mean Shift 向量逐步漂移到局部密度最大点并停止,从而达到跟踪目的。

\subsubsection{Mean Shift}

给定 $d$ 维空间中的 $n$ 个样本点 $x_i(i=1, \ldots, n)$,在 $x$ 点的 Mean Shift 向量的基本形式定义为:
\vspace{0.5cm}

\begin{equation}
M_h(x)=\left(\frac{1}{k} \sum_{x_i \in S_h} x_i\right)-x=\frac{1}{k} \sum_{x_i \in S_h}\left(x_i-x\right)
\vspace{0.5cm}
\end{equation}

其中, $S_h$ 是一个半径为 $h$ 的高维球区域, $k$ 表示 $n$ 个样本点中有 $k$ 个点落入区域 $S_h$ 中。直观地, Mean Shift 向量表示区域中 $k$ 个样本点相对于点 $x$ 求偏移向量再平均。该向量指向概率密度梯度的方向。

\begin{figure}[!ht]
	\center
	\includegraphics[width=0.65\linewidth]{fig2.png}
	\caption{Mean Shift 示意图}
	\label{fig:meanshift}
\end{figure}

如图 \ref{fig:meanshift} 所示,大圆所圈定的范围即为 $S_h$,小圆代表落入 $S_h$ 区域内的样本点 $x_i \in S_h$,中心的黑色实心圆即为 Mean Shift 的基准点 $x$,箭头表示样本点相对于基准点 $x$ 的偏移向量。可以看出,平均的偏移向量 $M_h(x)$ 会指向样本分量最多的区域,也就是概率密度函数的梯度方向。
\vspace{0.5cm}

\begin{equation}
M_h(x)=\frac{1}{k} \sum_{x_i \in S_h}\left(x_i-x\right)
\vspace{0.5cm}
\end{equation}

\subsubsection{核函数}
为了解决所有样本点对均值平移向量的贡献相同的问题,在 Mean Shift 的基础上引入两个参数,即核函数和权重。核函数的定义为 $x \in R^D,\|x\|^2=x^T x$,若函数 $K(𝑥)$ 存在一个剖面函数: $k:[0, \infty) \rightarrow R$, 即 $K(x)=k\left(\|x\|^2\right)$,并且 $k(𝑟)$ 满足:1)非负的;2)非增的;3)分段连续的,且 $\int_0^{\infty} k(r) d r<\infty$,那么函数 $K(𝑥)$ 就被称为核函数。常用的核函数包括 Epanechikov 核函数、均匀核函数和高斯核函数。本文采用的是 Epanechikov 核函数,即计算空间任意一点到中心位置的欧氏距离。其表达式如下:
\vspace{0.5cm}

\begin{equation}
K_E(\mathbf{x})=\left\{\begin{array}{cr}
c\left(1-\|\mathbf{x}\|^2\right) & \|\mathbf{x}\| \leq 1 \\
0 & \text { otherwise }
\end{array}\right.
\vspace{0.5cm}
\end{equation}

在跟踪问题中,核函数起到了位置加权的作用,即更加相信中心位置像素点的信息。

\subsubsection{均值漂移目标跟踪算法}

基于均值漂移的目标跟踪算法通过分别计算目标区域和候选区域内像素的特征值概率得到关于目标模型和候选模型的描述,然后利用相似函数度量初始帧目标模型和当前帧的候选模版的相似性,选择使相似函数最大的候选模型并得到关于目标模型的 Mean Shift 向量,这个向量正是目标由初始位置向正确位置移动的向量。由于均值漂移算法的快速收敛性,通过不断迭代计算 Mean Shift 向量,算法最终将收敛到目标的真实位置,达到跟踪的目的。流程如下:

(1)目标核函数直方图:
\vspace{1.1cm}

\begin{equation}
\vspace{0.5cm}
\hat{q}_u=C \sum_{i=1}^n \tikzmarknode{x}{\highlight{red}{$k\left(\left\|x_i\right\|^2\right)$}}\tikzmarknode{s}{\highlight{blue}{$\delta\left[b\left(x_i\right)-u\right]$}}.
\end{equation}
\begin{tikzpicture}[overlay,remember picture,>=stealth,nodes={align=left,inner ysep=1pt},<-]
    % For "X"
    \path (x.north) ++ (0,2em) node[anchor=south east,color=red!67] (scalep){\textbf{核函数,中心像素权重更大}};
    \draw [color=red!87](x.north) |- ([xshift=-0.3ex,color=red]scalep.south west);
    % For "S"
    \path (s.south) ++ (0,-1.5em) node[anchor=north west,color=blue!67] (mean){\textbf{统计直方图}};
    \draw [color=blue!57](s.south) |- ([xshift=-0.3ex,color=blue]mean.south east);
\end{tikzpicture}


其中,$C=1 / \sum_{i=1}^n k\left(\left\|x_i\right\|^2\right)$ 为归一化常数,$k$ 表示核函数(在核估计中通常是平滑作用), 目标区域共 $n$ 个像素点 $\left\{x_i\right\}_{i=1, \ldots, r}$,该区域颜色分布离散成 $m$ 级,$b\left(x_i\right)$ 表示像素点 $x_i$ 的量化值。 

(2)在当前帧,计算候选(待跟踪目标)核函数直方图:
\vspace{0.5cm}
\begin{equation}
\hat{p}_u(y)=C_h \sum_{i=1}^{n_h} k\left(\left\|\frac{y-x_i}{h}\right\|^2\right) \delta\left[b\left(x_i\right)-u\right]
\vspace{0.5cm}
\end{equation}

其中,$C_h=1 / \sum_{i=1}^n k\left(\left\|\frac{y-x_i}{h}\right\|^2\right)$,候选目标区域 $\left\{x_i\right\}_{i=1, \ldots, n_h}$ ,该区域的中心位置为 $y$,$h$ 表示核函数 $k$ 的窗宽。其余变量物理意义同上。

(3)计算候选目标与初始目标的相似度:

\vspace{0.5cm}
\begin{equation}
\hat{\rho}(y)=\sum_{u=1}^m \sqrt{\hat{p}_u(y) \hat{q}_u}
\vspace{0.5cm}
\end{equation}

进行 Taylor 展开,将 $\widehat{p}_u(y)$ 代入并化简得到:

\vspace{0.5cm}

\begin{equation}
\hat{\rho}(y) \approx \frac{1}{2} \sum_{u=1}^m \sqrt{\hat{p}_u\left(y_0\right) \hat{q}_u}+\frac{C_h}{2} \sum_{i=1}^{n_h} \mathrm{w}_i k\left(\left\|\frac{y-x_i}{h}\right\|^2\right)
\vspace{0.5cm}
\end{equation}

其中,$\mathrm{w}_i=\sum_{u=1}^m \sqrt{\frac{\hat{q}_u}{\hat{p}_u\left(\hat{y}_0\right)}} \delta\left[b\left(x_i\right)-u\right]$,$y_0$ 是 Mean Shift 迭代的起点位置, 在跟踪中通常是目标在上一帧的位置。由于第一项为常量,因此我们最大化 $\hat{\rho}(y)$ ,本质上寻找新的质心 $y$ 使得候选区域和模板的相似程度最大化。

(4)计算权值 $\{W\}_{i=1,2, \ldots, m}$

(5)利用 Mean Shift 方法求解目标的新位置 $y_1$: 
\vspace{0.5cm}

\begin{equation}
y_1=\frac{\sum_{i=1}^{n_h} x_i w_i g\left(\left\|\frac{y_0-x_i}{h}\right\|^2\right)}{\sum_{i=1}^{n_h} w_i g\left(\left\|\frac{y_0-x_i}{h}\right\|^2\right)}
\vspace{0.5cm}
\end{equation}


代码实现如下:

\vspace{0.3cm}
\lstinputlisting[language=Python,firstline=47,lastline=120]{main.py}

\subsection{基于分类思想的目标跟踪}

如前文所述,基于分类思想的目标跟踪将目标和背景信息同时考虑在内,其基本原理是把跟踪看作分类问题,通过训练分类器来区分背景和目标。具体来讲就是是把目标跟踪看作二分类问题,在线训练作为整体的多个弱分类器用来区分目标和背景。使用 AdaBoost 把作为整体的多个弱分类器合并为一个强的分类器,该分类器用于下一帧的分类,区分像素属于目标还是背景,并得出置信图,并在置信图上利用 Mean Shift 算法找出峰值点也就是目标的位置。在跟踪过程中通过在线训练新的弱分类器并加入到分类器集合里从而在连贯时间上保持更新弱分类器这一整体。

首先需要在第一帧将目标的位置进行手工标注,对于弱分类器而言,所有像素都是潜在的训练数据。为了减少训练时间,本文根据视频训练中目标的大小,将标注目标及其周围的一部分像素(背景)作为训练数据。每个像素都使用一个 11 维的特征向量来描述(由 8 维的 $5\times 5$ 邻域局部方向直方图和 3 维的像素颜色组成),来自目标和背景的像素对应的标签分别为 1 和 -1,即
\vspace{0.3cm}
\begin{equation}
y_i= \begin{cases}+1 & \text { inside }\left(r_j\right) \\ -1 & \text { otherwise }\end{cases}
\vspace{0.3cm}
\end{equation}

其中,$r_j$ 为当前帧的目标位置。

本文使用支持向量机作为弱分类器。对于每一个弱分类器,可以计算其误差率,计算方法为:
\vspace{0.3cm}

\begin{equation}
e r r=\sum_{i=1}^N w_i\left|h_t\left(x_i\right)-y_i\right|
\vspace{0.3cm}
\end{equation}

其中,$w_i$ 是像素的权重,为像素被选择训练的概率。$h_t\left(x_i\right)$ 是像素 $x_i$ 的类别,而 $y_i$ 为其“正确”的类别标签。在初始状态下,所有的像素权重都是相等的,并随着训练过程而不断更新,如果像素被错误分类,则像素权重会随之增加。也就是说,在随后的训练集中,难以分类的像素更有可能被选择。权重的更新公式为:
\vspace{0.3cm}

\begin{equation}
w_i=w_i \exp \left\{\alpha_t\left|h_t\left(x_i\right)-y_i\right|\right\}
\vspace{0.3cm}
\end{equation}


集成学习技术把若干弱分类器组合成一个强的分类器。如本文使用的 AdaBoost 每次在更难的样本上训练一个弱分类器加入到强分类器使得最终的分类器比任何一个弱分类器都好。在跟踪过程中一直更新弱分类器集合,将目标从背景中分离。因此,该方法并不准确的描述目标,而是使用分类器集合来决定一个像素属于目标还是背景,从而可以更好地应对光照、尺寸的变化,目标的变形和遮挡等问题。在集成为强分类器时,需要为每个若分类器赋予一个权重,性能较好的分类器则具有较高的权重,该权重是由弱分类器的误差率决定的:
\vspace{0.3cm}
\begin{equation}
\alpha_t=\frac{1}{2} \log \frac{1-e r r}{e r r}
\vspace{0.3cm}
\end{equation}

其算法流程如算法 \ref{ensemble} 所示。

 \begin{algorithm}
        \caption{Ensemble Tracking}
         \label{ensemble}
        \LinesNumbered
        \KwIn{n video frames $I_1, \cdots, I_n$, Rectangle $r_1$ of object in first frame}
        \KwOut{Rectangles $r_2 , \cdots, r_n$}
        
        Initialization (for frame $I_1$):\\
        Extract $\{x_i\}^N_i=1$ examples with labels $\{y_i\}^N_i=1$\\
        Initialize weights $\{w_i\}^N_i=1$ to be $\frac{1}{N}$\\
        \For{$t = 1 \to T$} 
        {
        (a) Make $\{w_i\}^N_i=1$ a distrubition\\
        (b) Train weak classifier $h_t$\\
        (c) Set $\operatorname{err}=\sum_{i=1}^N w_i\left|h_t\left(\mathbf{x}_{\mathbf{i}}\right)-y_i\right|$\\
        (d) Set weak classifier weight $\alpha_t=\frac{1}{2} \log \frac{1-e r r}{e r r}$\\
        (e) Update example weights $w_i=w_i e^{\left(\alpha_t\left|h_t\left(\mathbf{x}_{\mathbf{i}}\right)-y_i\right|\right)}$\\
        }
        
        The strong classifier is given by $\operatorname{sign}(H(\mathbf{x}))$ where $H(x)=\sum_{t=1}^T \alpha_t h_t(\mathbf{x})$\\
        
         \For{each new frame $I_j$} 
         {
         Extract $\left\{\mathbf{x}_{\mathbf{i}}\right\}_{i=1}^N$ examples\\
         Test the examples using the strong classifier $H(x)$ and create confidence image $L_j$\\
         Run mean-shift on $L_j$ with $r_{j-1}$ as the initial guess. Let $r_j$ be the result of the mean shift algorithm\\
         Define labels $\left\{y_i\right\}_{i=1}^N$ with respect to the new rectangle $r_j$\\
         Remove $K$ oldest weak classifiers\\
         Initialize weights $\left\{w_i\right\}_{i=1}^N$ to be $\frac{1}{N}$\\
         \For(\tcp*[f]{Update weights}){$l=K+1 \to T$}
         {
         (a) Make $\left\{w_i\right\}_{i=1}^N$ a distribution\\
         (b) Choose $h_t(\mathbf{x})$, with minimal error $e r r$, from $\left\{h_{K+1}(\mathbf{x}), \ldots, h_T(\mathbf{x})\right\}$\\
         (c) update $\alpha_t$ and $\left\{w_i\right\}_{i=1}^N$\\
         (d) Remove $h_t(\mathbf{x})$ from $\left\{h_{K+1}(\mathbf{x}), \ldots, h_T(\mathbf{x})\right\}$\\
         }
         
         \For(\tcp*[f]{Add new weak classifiers}){$l=1 \to K$}
         {
         (a) Make $\left\{w_i\right\}_{i=1}^N$ a distribution\\
         (b) Train weak classifier $h_t$\\
         (c) Compute $e r r$ and $\alpha_t$\\
         (d) Update example weights $\left\{w_i\right\}_{i=1}^N$
         }
         
         The updated strong classifier is given by $\operatorname{sign}(H(\mathbf{x}))$ where $H(x)=\sum_{t=1}^T \alpha_t h_t(\mathbf{x})$
         }
    \end{algorithm}



代码实现如下\footnote{特征提取及标签生成等相关代码由于篇幅原因在此不再展示}:

\vspace{0.3cm}
\lstinputlisting[language=Python,firstline=114,lastline=197]{main2.py}

\section{实验}
\label{sec:experiment}

\begin{figure*}[!ht]
 \centering
  \begin{minipage}[b]{\linewidth} 	
  \subfloat[]{
    \begin{minipage}[b]{0.5\linewidth} 
      \centering
      \includegraphics[width=\linewidth]{glcm_feats1}
       \end{minipage}
  }
%  \newline
    \subfloat[]{
    \begin{minipage}[b]{0.5\linewidth}
      \centering
      \includegraphics[width=\linewidth]{glcm_feats2}
     \end{minipage}
  }
  \newline
   \subfloat[]{
    \begin{minipage}[b]{0.5\linewidth}
      \centering
       \includegraphics[width=\linewidth]{glcm_feats3}
       \end{minipage}
  }
    \subfloat[]{
    \begin{minipage}[b]{0.5\linewidth}
      \centering
      \includegraphics[width=\linewidth]{glcm_feats4}
     \end{minipage}
  }
  \end{minipage}
  \vfill
  \caption{灰度共生矩阵的统计特征可视化结果:(a) Texture\_mosaic\_1, (b) Texture\_mosaic\_2, (c)Texture\_mosaic\_3, (d) Texture\_mosaic\_4;第一行:纹理图像、Mean、Variance、Entropy、Energy;第二行:Inertia、Correlation、Autocorrelation、Contrast、Dissimilarity.}
  \label{fig:glcm_feats}
	
\end{figure*}

\begin{table*}[!htbp]
\caption{基于灰度共生矩阵的统计量及不同合成纹理图像指标选取示意表}
\centering
\label{tab:measurement}
\setlength{\tabcolsep}{3.5mm}{
\begin{tabular}{c|ccccccccc}
\hline
& Mean & Variance & Entropy & Energy & Inertia & Correlation & Autocorrelation & Contrast & Dissimilarity\\
\hline
1 & \checkmark & \checkmark & \checkmark & \checkmark & & \checkmark & & \checkmark &\\
2 & \checkmark & \checkmark & \checkmark & \checkmark & \checkmark & \checkmark & & &\\
3 & & \checkmark & \checkmark &  & \checkmark & & \checkmark & \checkmark & \checkmark\\
4 & \checkmark & \checkmark &  &  & \checkmark & & \checkmark & \checkmark & \checkmark\\
\hline
\end{tabular}}
\end{table*}

\subsection{灰度共生矩阵法的纹理分割实验}

如前文所述,在采用灰度共生矩阵法对纹理图像进行分割时,拟在表 \ref{tab:measurement} 中列出的统计量中进行选择用以对纹理进行表示和度量。因此本文首先先对测试图像的统计量结果进行了可视化处理,如图 \ref{fig:glcm_feats} 所示。在此基础上,根据不同统计量的可视化结果分别对测试图像使用的统计量进行选择,选取结果如表 \ref{tab:measurement} 所示。

在固定窗口大小为19,滑动步长为1,灰度等级为16的前提下,按照选取的统计量逐步计算各图像的灰度共生矩阵及相关统计量,并采用 K-Means 聚类方法对纹理图像进行分割,分割结果如图 \ref{fig:seg} 所示。

\begin{figure*}[!ht]
 \centering
  \begin{minipage}[b]{\linewidth} 	
  \subfloat[]{
    \begin{minipage}[b]{0.25\linewidth} 
      \centering
      \includegraphics[width=\linewidth]{Texture_mosaic_1}
       \end{minipage}
  }
    \subfloat[]{
    \begin{minipage}[b]{0.25\linewidth}
      \centering
      \includegraphics[width=\linewidth]{Texture_mosaic_2}
     \end{minipage}
  }
   \subfloat[]{
    \begin{minipage}[b]{0.25\linewidth}
      \centering
       \includegraphics[width=\linewidth]{Texture_mosaic_3}
       \end{minipage}
  }
    \subfloat[]{
    \begin{minipage}[b]{0.25\linewidth}
      \centering
      \includegraphics[width=\linewidth]{Texture_mosaic_4}
     \end{minipage}
  }
  \end{minipage}
  \vfill
  \caption{灰度共生矩阵的纹理分割结果:(a) Texture\_mosaic\_1, (b) Texture\_mosaic\_2, (c)Texture\_mosaic\_3, (d) Texture\_mosaic\_4.}
  \label{fig:seg}
	
\end{figure*}

分割结果受到窗口大小、滑动步长及聚类初始中心点的选择等因素的影响,文本将会在对比实验部分选择其中的一些因素对分割结果的影响进行探讨。

\subsection{Gabor 滤波器的纹理分割实验}

固定参数 Gabor 滤波器核的大小为 19,高斯窗口大小为 31,高斯函数的纵横比为 0.5,高斯窗口函数的标准差为 7,用于聚类的行和列的空间权重为 2,在这一参数条件下,Gabor 滤波器的纹理分割结果如图 \ref{fig:seg2} 所示。


\begin{figure*}[!ht]
 \centering
  \begin{minipage}[b]{\linewidth} 	
  \subfloat[]{
    \begin{minipage}[b]{0.25\linewidth} 
      \centering
      \includegraphics[width=\linewidth]{Texture_mosaic_1_gabor}
       \end{minipage}
  }
    \subfloat[]{
    \begin{minipage}[b]{0.25\linewidth}
      \centering
      \includegraphics[width=\linewidth]{Texture_mosaic_2_gabor}
     \end{minipage}
  }
   \subfloat[]{
    \begin{minipage}[b]{0.25\linewidth}
      \centering
       \includegraphics[width=\linewidth]{Texture_mosaic_3_gabor}
       \end{minipage}
  }
    \subfloat[]{
    \begin{minipage}[b]{0.25\linewidth}
      \centering
      \includegraphics[width=\linewidth]{Texture_mosaic_4_gabor}
     \end{minipage}
  }
  \end{minipage}
  \vfill
%  \vspace{5mm}
  \caption{Gabor 滤波器的纹理分割结果:(a) Texture\_mosaic\_1, (b) Texture\_mosaic\_2, (c)Texture\_mosaic\_3, (d) Texture\_mosaic\_4.}
  \label{fig:seg2}	
\end{figure*}

  对比图 \ref{fig:seg} 来看,与灰度共生矩阵的纹理分割结果相比,如果忽略聚类初始中心点选择的影响,Gabor 滤波器的纹理分割结果在不同纹理的交界处的分割效果更好,这是由于其提取目标的局部空间和频率域信息方面具有良好的特性,且 Gabor 小波对于图像的边缘敏感,能够提供良好的方向选择和尺度选择特性,对噪声更加鲁棒。

\subsection{对比实验}

这一章节将在测试图片 Texture\_mosaic\_3 上对不同的参数进行调整,进行对比实验。为消除聚类初始中心点选择对分割结果的影响,根据图像特性,选择图像的四个四分位点作为聚类的初始中心。下面分别对滑动窗口大小和步长对分割结果的影响进行讨论分析。

\subsubsection{滑动窗口大小的影响}

在探究滑动窗口大小对纹理分割结果的影响时,固定其他参数不变,窗口大小分别选取 13、19、30和50,分割结果如图 \ref{fig:winsize_result} 所示。


\begin{figure}[!ht]
	\vspace{-0.8cm}
  \centering
  \begin{minipage}[b]{\linewidth} 	
  \subfloat[]{
    \begin{minipage}[b]{0.25\linewidth} 
      \centering
      \includegraphics[width=\linewidth]{win_size_13}
       \end{minipage}
  }
    \subfloat[]{
    \begin{minipage}[b]{0.25\linewidth}
      \centering
      \includegraphics[width=\linewidth]{Texture_mosaic_3}
     \end{minipage}
  }
   \subfloat[]{
    \begin{minipage}[b]{0.25\linewidth}
      \centering
      \includegraphics[width=\linewidth]{win_size_30}
       \end{minipage}
  }
	\subfloat[]{
    \begin{minipage}[b]{0.25\linewidth}
      \centering
      \includegraphics[width=\linewidth]{win_size_50}
       \end{minipage}
  }

  \end{minipage}
  \vfill
  \caption{不同大小滑动窗口得到纹理分割结果:(a) $win\_size=13$, (b) $win\_size=19$, (c) $win\_size=30$, (d) $win\_size=50$.}
  \label{fig:winsize_result}
\end{figure}

从图 \ref{fig:winsize_result} 可以看出,当滑动窗口的选择偏小时,在区域内部会有被错分的较小区域,其原因可能是窗口的选择过小,导致对纹理特征的估计发生了错误。而随着窗口的增大,区域内部被错分的情况消失,但随之出现的是区域的交界处易被划分错误,其原因可能是窗口的选择过大,不能很好地捕捉到纹理的局部特性。

\subsubsection{滑动步长大小的影响}

在探究滑动步长大小对纹理分割结果的影响时,固定其他参数不变,步长大小分别选取 1、2和3,分割结果如图 \ref{fig:stride_result} 所示。

\begin{figure}[!ht]
	\vspace{-0.8cm}
  \centering
  \begin{minipage}[b]{\linewidth} 	
  \subfloat[]{
    \begin{minipage}[b]{0.3\linewidth} 
      \centering
      \includegraphics[width=\linewidth]{Texture_mosaic_3}
       \end{minipage}
  }
    \subfloat[]{
    \begin{minipage}[b]{0.3\linewidth}
      \centering
      \includegraphics[width=\linewidth]{stride_2}
     \end{minipage}
  }
   \subfloat[]{
    \begin{minipage}[b]{0.3\linewidth}
      \centering
      \includegraphics[width=\linewidth]{stride_3}
       \end{minipage}
  }
  \end{minipage}
  \vfill
  \caption{不同大小滑动步长得到纹理分割结果:(a) $stride=1$, (b) $stride=2$, (c) $stride=3$.}
  \label{fig:stride_result}
\end{figure}

从图 \ref{fig:stride_result} 可以看出,随着滑动步长的逐渐增大,分割边界逐渐变得不再平滑,呈现出锯齿状。


\section{总结}
本文主要对目标跟踪的两种方法进行了探讨和实验,即基于均值的目标跟踪方法和基于分类思想的目标跟踪方法。前者计算量不大,在目标区域已知的情况下完全可以做到实时跟踪,但相对而言对边缘遮挡、目标旋转、变形和背景运动不敏感。而后者将背景信息考虑在内,将目标跟踪看作分类问题,把耗费时间的训练阶段分解成一系列简单的可以在线的学习计算任务,同时自动调整了不同分类器的权值,在不同的特征值空间训练,可以保持稳定的变化来适应光照和遮挡等。但由于粗糙的特征标注和缺乏尺度的变化,在实际实验中稳定性和实时性还有待进一步提高。

% Can use something like this to put references on a page
% by themselves when using endfloat and the captionsoff option.
\ifCLASSOPTIONcaptionsoff
  \newpage
\fi



%{\small
%\bibliographystyle{IEEEtran}
%\bibliography{reference/egbib}
%}

% that's all folks
\end{sloppypar}
\end{document}


